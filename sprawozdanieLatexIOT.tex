\documentclass{article}
\usepackage[utf8]{inputenc}
\usepackage{xcolor}
\title{Sprawozdanie}
\author{konrad.nowakowski1911 }
\date{October 2018}

\usepackage{natbib}
\usepackage{graphicx}
\pagecolor{purple!30}
\begin{document}

\maketitle
\chapter{Inforamacje o studencie}
    \begin{table}[ht]
        \center
        \scalebox{1.3}{
        \begin{tabular}{|c|c|c|c|}
        \hline
         Imie oraz  Nazwisko & Przedmiot & Wydzial & Grupa \\   
        \hline
        Konrad Nowakowski & IOT & WEAiI & 3ID15A  \\
        \hline
        \end{tabular}}
        \label{tab:my_label}
    \end{table}




\section{Kolorowanie tekstu}
{\color{red}Lorem ipsum dolor sit amet, consectetur adipiscing elit. Maecenas eu ultrices nunc, vel accumsan ex. Nullam vulputate lobortis erat id euismod. Cras scelerisque efficitur bibendum. Proin facilisis tortor sit amet eros tempus volutpat. Curabitur vitae enim tincidunt, gravida nisl eget, molestie eros. Duis posuere a nibh sit amet gravida. Lorem ipsum dolor sit amet, consectetur adipiscing elit.}

{\color{blue}feugiat aliquam risus. Morbi tempor tempus velit at vulputate. Donec est nisl, ullamcorper euismod viverra lacinia, dignissim sit amet augue. Quisque dignissim nunc id ornare bibendum. Sed luctus consequat lorem, at bibendum nibh iaculis sed. Aenean laoreet sem in arcu sollicitudin, ac pretium tortor vulputate. Proin porta elit eu leo imperdiet, at auctor neque tempus. Nulla nec turpis vitae arcu accumsan consectetur. Maecenas dui est, efficitur a vestibulum eget, sollicitudin ut diam.}



\section{Rożne formatowanie czcionek}
\tiny{to jest malautka czcionka}\\
\small {to jest mala czcionka} \\
\large {to jest ogrmona czcionka}\\
\Large {to jest jeszcze wieksza czcionka od large}

\section{Opis srodowiska}
Latex jest system skladu znakomicie nadajacym sie do tworzenia publikacji naukowych i technicznych o wysokiej jakosci typograficznej. Latex nadaje sie rowniez do przygotowania dowolnego rodzaju dokumentow, poczyniajac od prostych listow, a konczac na grubych ksiazkach. Do formatowania dokumentow Latex wykorzysuje program TEX. LaTeX ulatwia sklad tekstu pozwalajac autorowi skupic sie na tresci i strukturze tekstu.
Obecnie zwykle nie pisze sie tekstu zrodlowego w "czystym" TeX-u (plain TeX), uzywa sie LaTeX-a wraz z dodatkowymi pakietami okreslanymi mianem klas. Klasy ulatwiaja prace nad wyspecjalizowanymi rodzajami dokumentow - na przyklad publikacjami zawierajacymi rozbudowane wzory matematyczne lub chemiczne. Ponadto, dla ulatwienia wspolpracy z autorami artykulow, czasopisma skladane w LaTeX-u moga dostarczac wlasne wyspecjalizowane klasy.(opis ten został zaczerpniety z pracy "Nie za krótkie wprowadzenie do systemów Latex2 autorow: Huber Partl, Irena Hyna, Elisabeth Schlegl oraz strony internetowej wikipedia.org)

\cite{Oetiker:wprowadzenie:Latex}
\bibliographystyle{ieeetr}
\bibliography{references}




\end{document}


